\graphicspath{{detail_design/fig/}}

\chapter{Detailed Design}
\label{chap:detail_design}

According to the electrical characteristics of the major components (shown in table \ref{tab:electrical_chars}), the entire system can be powered by a 6V source. Voltage regulators will be used to provide 5V to the microcontroller and 3.3V to the sensors. 
\\
The microcontroller uses a 5V supply to maximize the voltage output of the pins (this will make designing the pump driver circuit easier - the driver circuit is going to need like 7+ volts)

\begin{table}[!h]
\centering
\caption{Electrical characteristics of major components.}
\label{tab:electrical_chars}
    \begin{tabular}{|c||c|c||c|c|} 
        \hline
        Component & \multicolumn{2}{c||}{Input} & \multicolumn{2}{c|}{Output} \\
       %\hline
         & Voltage [V] & Current [mA] & Voltage [V] & Current [mA] \\
        \hline
        \hline
        ESP32 \cite{esp_datasheet} & 3.3 & 500 & & \\
         & 5 & & & \\
        \hline
        ESP32 pins \cite{esp_datasheet} & 0.75 * $V_{DD}$ & & 0.8 * $V_{DD}$ & 40\\
        & $V_{DD}$ + 0.3 & & & \\
        \hline
        Soil moisture sensor \cite{Moisture_sensor_datasheet} & 3.3 & $\pm$ 5 \tablefootnote{Based off characteristics of similar sensors} \cite{Moisture_sensor_current} & 1.2 & \\
        & 5.5 & & 2.5 & \\
        \hline
        UV light sensor \cite{UV_sensor_datasheet} & 3 & 100 \tablefootnote{Maximum rating based off characteristics of similar sensors} & 0 & \\
        & 5 & & 3 & \\
        \hline
        Pump \cite{pump_datasheet} & 6 & 1800 (2500 peak) \cite{pump_datasheet_similar} \tablefootnote{Based off characteristics of similar pump} & & \\
        \hline
    \end{tabular}
\end{table}
%%%%%%%%%%%%%%%%%%%%%%%%%%%%%%%%%%%%%%%%
\section{UV light exposure and soil moisture level measurement and logging}
Due to the current requirement of the UV light sensor it will need to be powered from the main power source as the microcontroller pins cannot supply enough current to meet the maximum rating. 
\\
Power will therefore be supplied to components using 5V and 3.3V voltage regulators as described in section \ref{sec:battery}. To prevent exceeding the maximum current of the regulators, the sensors are connected to the 3.3V supply and the microcontroller to the 5V supply. 

%%%%%%%%%%%%%%%%%%%%%%%%%%%%%%%%%%%%%%%%
\section{Automatic watering}
The pump requires 6V and has a peak current of approximately 2.5A and a rated power of 5W (based on the rating of a similar pump). To accommodate the maximum potential power draw of 15W, a TIP31C npn power transistor will be used alongside a 2N2222A npn transistor in a Darlington pair. 
\\
The TIP31C has a maximum rated collector current of 3A and can dissipate up to 40W. The operating current of the pump at 5W, 6V is 0.8A. At \(I_C = 0.8A\), the TIP31C has a turn on voltage \(V_{BE\_on}\) of approximately 0.9V and a DC current gain \(\beta\) of 43. The emitter current of the 2N2222A transistor can be calculated using \(I_C\) and \(\beta\) of the TIP31C: 
\[I_{E(2N2222A)} = I_{B(TIP31C)} = \frac{I_{C(TIP31C)}}{\beta_{TIP31C}} = \frac{0.8}{43} = 18.60mA\]
The 2N2222A will therefore have a turn on voltage \(V_{BE\_on}\) of 0.7V and a DC current gain \(\beta\) of 225. 
\\

Due to the turn on voltages, the pump control circuit requires a control voltage of 7.6V to deliver 6V to the pump. 
%%%%%%%%%%%%%%%%%%%%%%%%%%%%%%%%%%%%%%%%
\section{App}
\subsection{Wifi vs bluetooth for data transfer}

%%%%%%%%%%%%%%%%%%%%%%%%%%%%%%%%%%%%%%%%
\section{Battery and charging}
\label{sec:battery}
\subsection{battery options}

%%%%%%%%%%%%%%%%%%%%%%%%%%%%%%%%%%%%%%%%
\section{PCB}

%%%%%%%%%%%%%%%%%%%%%%%%%%%%%%%%%%%%%%%%
\section{Case}
